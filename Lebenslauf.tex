% Dokumententyp
\documentclass[letterpaper,12pt]{article}

\usepackage{latexsym}
\usepackage[empty]{fullpage}
\usepackage{titlesec}
\usepackage{marvosym}
\usepackage[usenames,dvipsnames]{color}
\usepackage{verbatim}
\usepackage{enumitem}
\usepackage[hidelinks]{hyperref}
\usepackage{fancyhdr}
\usepackage[ngerman]{babel}
\usepackage{tabularx}
\usepackage{graphicx}
\usepackage[export]{adjustbox}
\input{glyphtounicode}


%%%%% Allgemeine Einstellungen %%%%
\pagestyle{fancy}
\fancyhf{}
\fancyfoot{}
\renewcommand{\headrulewidth}{0pt}
\renewcommand{\footrulewidth}{0pt}

\addtolength{\oddsidemargin}{-0.5in}
\addtolength{\evensidemargin}{-0.5in}
\addtolength{\textwidth}{1in}
\addtolength{\topmargin}{-.5in}
\addtolength{\textheight}{1.0in}

\urlstyle{same}

\raggedbottom
\raggedright
\setlength{\tabcolsep}{0in}


\titleformat{\section}{
  \vspace{-4pt}\scshape\raggedright\large
}{}{0em}{}[\color{black}\titlerule \vspace{-5pt}]


\pdfgentounicode=1

\newcommand{\resumeItem}[1]{
  \item\small{
    {#1 \vspace{-2pt}}
  }
}

\newcommand{\resumeSubheading}[4]{
  \vspace{-2pt}\item
    \begin{tabular*}{0.98\textwidth}[t]{l@{\extracolsep{\fill}}r}
      \textbf{#1} & #2 \\
      \textit{\small#3} & \textit{\small #4} \\
    \end{tabular*}\vspace{-7pt}
}

\newcommand{\resumeSubSubheading}[2]{
    \item
    \begin{tabular*}{0.98\textwidth}{l@{\extracolsep{\fill}}r}
      \textit{\small#1} & \textit{\small #2} \\
    \end{tabular*}\vspace{-7pt}
}

\newcommand{\resumeProjectHeading}[2]{
    \item
    \begin{tabular*}{0.98\textwidth}{l@{\extracolsep{\fill}}r}
      \small#1 & #2 \\
    \end{tabular*}\vspace{-7pt}
}

\newcommand{\resumeSubItem}[1]{\resumeItem{#1}\vspace{-4pt}}

\renewcommand\labelitemii{$\vcenter{\hbox{\tiny$\bullet$}}$}

\newcommand{\resumeSubHeadingListStart}{\begin{itemize}[leftmargin=0.15in, label={}]}
\newcommand{\resumeSubHeadingListEnd}{\end{itemize}}
\newcommand{\resumeItemListStart}{\begin{itemize}}
\newcommand{\resumeItemListEnd}{\end{itemize}\vspace{-5pt}}

%----------

\begin{document}
\vfill
\begin{minipage}{0.15\linewidth}
    \includegraphics[width=\linewidth]{Bild.jpg} % Bewerbungsfoto muss im selben Verzeichnis sein
\end{minipage}\hfil
\begin{minipage}{0.68\linewidth}
\begin{center}
      \textbf{\Huge \scshape Max Mustermann} \\ \vspace{1pt}
    \small Anschrift: Musterstraße 1; 81111 Augsburg $|$
    \small geb. 1. Mai 1985 in München $|$
    \small Mobil: 0111/111111111 $|$ 
    \href{mailto:x@x.com}{E-Mail: max.mustermann@gmail.com} $|$   
\end{center}
\end{minipage}


%-----------Ausbildung-----------
% ggf. neue Sektion einfügen
\section{Ausbildung}
  \resumeSubHeadingListStart
  
    \resumeSubheading
      {Schulausbildung am Gymnasium}{Musterstadt, DE}
      {Allgemeine Hochschulreife, Abschluss im Jahr 2018}{September 2010 - Juli 2018}
    \resumeSubheading
      {Studium an der Technischen Universität}{Musterstadt, DE}
      {Bachelor of Science, Ohne Abschluss}{Oktober 2017 - März 2018}
  \resumeSubHeadingListEnd


%-----------Berufliche Laufbahn-----------
% ggf. neue Sektion einfügen
\section{Berufliche Laufbahn}
  \resumeSubHeadingListStart

	\resumeSubheading
      {Angestellter bei ....}{August 2013 - Juli 2017}
      {......}{Musterstadt, DE}
      \resumeItemListStart
        \resumeItem{Aufgaben: .......}
      \resumeItemListEnd
      
    \resumeSubheading
      {......}{Dezember 2010 - Oktober 2018}
      {......}{Musterstadt, DE}
      \resumeItemListStart
        \resumeItem{Aufgaben: ......}
      \resumeItemListEnd

    \resumeSubheading
      {......}{August 2018 - Oktober 2018}
      {......}{Musterstadt, DE}
      \resumeItemListStart
        \resumeItem{Aufgaben: ......}
    \resumeItemListEnd

    \resumeSubheading
      {......}{Juni 2020 - Oktober 2020}
      {......}{Musterstadt, DE}
      \resumeItemListStart
        \resumeItem{Aufgaben: ......}
      \resumeItemListEnd
      
  \resumeSubHeadingListEnd


%
%-----------Kenntnisse und Fähigkeiten-----------
\section{Kenntnisse und Fähigkeiten}
 \begin{itemize}[leftmargin=0.15in, label={}]
    \small{\item{
     \textbf{Sprachen}{: Englisch: Level C1, Französisch: Level B1, Spanisch: Level A1} \\
     \textbf{PC-Kenntnisse}{: Sehr gute Microsoft Office - Kenntnisse (Word,
PowerPoint, Excel), Kenntnisse mit Statistik-Programmen(R und SPSS) , LaTeX und Java - Kenntnisse} \\
     \textbf{Führerschein}{: Klasse B} \\
    }}
 \end{itemize}


%----------
\end{document}
